%%%%%%%%%%%%%%%%%%%%%%%%%%%%%%%%%%%%%%%%%%%%%%%%%%%%%%%%%%%%%
\chapter*{ΕΙΣΑΓΩΓΗ}
\addcontentsline{toc}{chapter}{ΕΙΣΑΓΩΓΗ}
%%%%%%%%%%%%%%%%%%%%%%%%%%%%%%%%%%%%%%%%%%%%%%%%%%%%%%%%%%%%%

%Εδώ η Γεωργία θα γράψει την εισαγωγή της. Στην πορεία θα συζητήσουμε
%την δομή της εισαγωγής~\cite{Kourounis2014}. 
%Πρέπει η εισαγωγή να έχει παρόμοιο θεματικό σκελετό με την~\cite{Jansen:book,Chen:2012} 
%περίληψη και τα συμπεράσματα στο τέλος της εργασίας.\\ \\
%Το αντικείμενο είναι λοιπόν η ανάλυση χρονοσειρών,
%δηλαδή η χρήση μεθόδων που θα μας επιτρέψουν να διερευνήσουμε το μηχανισμό
%(στοχαστική διαδικασία ή δυναμικό σύστημα) που παράγει τη χρονοσειρά, να
%εκτιμήσουμε χαρακτηριστικά του, να αναπτύξουμε μοντέλο για να το περιγράψουμε
%και να κάνουμε προβλέψεις της εξέλιξης του, δηλαδή τις επόμενες τιμές στη
%χρονοσειρά.




\bigskip

\begin{flushright}
\begin{minipage}{150pt}
Γ.\ Παπαδοπούλου, Χανιά 2015.
\end{minipage}
\end{flushright}



\endinput
%%% Local Variables: 
%%% mode: latex
%%% TeX-master: "ptyxiakn"
%%% End: 
