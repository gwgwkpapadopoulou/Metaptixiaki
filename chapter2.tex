
%%%%%%%%%%%%%%%%%%%%%%%%%%%%%%%%%%%%%%%%%%%%%%%%%
\chapter{ΜΕΘΟΔΟΙ ΠΡΟΒΛΕΨΗΣ}
%%%%%%%%%%%%%%%%%%%%%%%%%%%%%%%%%%%%%%%%%%%%%%%%%
\section{ ΚΡΙΤΗΡΙΑ ΑΞΙΟΛΟΓΗΣΗΣ ΤΩΝ ΜΕΘΟΔΩΝ ΠΡΟΒΛΕΨΗΣ}
Η ανάλυση χρονοσειρών (time series analysis) ασχολείται αποκλειστικά με τη
διερεύνηση της διαχρονικής συμπεριφοράς των τιμών μιας μεταβλητής, οι
παρατηρήσεις της οποίας προέρχονται από χρονοσειρά. Η πρόβλεψη των
μελλοντικών τιμών της μεταβλητής σύμφωνα με την ανάλυση χρονοσειρών μπορεί
να προέλθει απο διάφορες κατηγορίες μεθόδων προβλέψης, όπως η μέθοδος Εξομάλυνσης και η Διάσπαση των χρονοσειρών, με τις οποίες θα ασχοληθούμε στην παρούσα εργασία.\\
Για την επιλογή της κατάλληλης μεθόδου χρησιμοποιούνται τα κριτήρια
αξιολόγησης των μεθόδων προβλέψεων. Τα κριτήρια αυτά βασίζονται στις τιμές των
αποκλίσεων των προβλεπόμενων τιμών από τις αντίστοιχες πραγματικές τιμές της
χρονοσειράς.\\
Για μία μεταβλητή Y, η απόκλιση της προβλεπόμενης τιμής της $ \widehat{Y}_t $  από την
αντίστοιχη πραγματική τιμή της $Y_t$ για την περίοδο t, όπου $t=1,2,3,\ldots,n , \:$ ονομάζεται
σφάλμα της πρόβλεψης (forecast error), συμβολίζεται με $e_t$ και ορίζεται ως:
$\: e_t = Y_t - \widehat{Y}_t$\\
Η παραπάνω σχέση εκφράζει για κάθε περίοδο t τη διαφορά μεταξύ της πραγματικής
τιμής $Y_t$ και της αντίστοιχης προβλεπόμενης τιμής $ \widehat{Y}_t $ που προήλθε από τη μέθοδο
πρόβλεψης που χρησιμοποιήθηκε.\\
Επομένως, για να προσδιορίσουμε την αξιοπιστία μιας συγκεκριμένης μεθόδου
πρόβλεψης, θα πρέπει να μελετήσουμε τη διαχρονική συμπεριφορά των τιμών των
σφαλμάτων της πρόβλεψης. Αυτό γίνεται με την εφαρμογή διάφορων κριτηρίων,
σύμφωνα με τα οποία αξιολογούμε τη χρησιμοποιούμενη μέθοδο πρόβλεψης. Κάθε
ένα από τα κριτήρια αυτά ορίζεται από μία συγκεκριμένη συναρτησιακή σχέση των
σφαλμάτων της πρόβλεψης και μπορεί να χρησιμοποιηθεί όχι μόνο για την
αξιολόγηση μιας μεθόδου πρόβλεψης αλλά και για την επιλογή της “ καλύτερης ”
μεταξύ δύο ή περισσοτέρων εναλλακτικών μεθόδων προβλέψεων. Τα κριτήρια αυτά
είναι:\\
\begin{itemize}
\item \textbf{Μέση απόλυτη απόκλιση MAD} (Mean Absolute Deviation)\\
Η μέση απόλυτη απόκλιση ορίζεται ως το άθροισμα των απόλυτων τιμών του
σφάλματος της πρόβλεψης διαιρούμενο με τον αριθμό των περιόδων n, στις οποίες
έγιναν προβλέψεις, δηλαδή:\\
$$ MAD= \frac{1}{n} \sum_{t=1}^n \vert Y_t-\widehat{Y}_t \vert =\frac{1}{n} \sum_{t=1}^n \vert e_t \vert $$
Το MAD εκφράζει τη μέση τιμή των απολύτων αποκλίσεων των προβλεπόμενων
τιμών της χρονοσειράς από τις αντίστοιχες πραγματικές και έχει τα ακόλουθα
χαρακτηριστικά. Πρώτον, η μονάδα μέτρησης του είναι η ίδια με εκείνη των τιμών
της χρονοσειράς και έτσι είναι εύκολη η ερμηνεία του. Δεύτερον, στον υπολογισμό
του λαμβάνονται υπ’ όψιν μόνο οι απόλυτες τιμές των σφαλμάτων και όχι οι
πραγματικές τιμές τους. Αυτό σημαίνει ότι το MAD είναι ανεξάρτητο από θετικές ή αρνητικές τιμές του σφάλματος, δηλαδή είναι ανεξάρτητο από το αν οι τιμές των
προβλέψεων είναι μικρότερες ή μεγαλύτερες των
πραγματικών τιμών.Και τέλος, το MAD βασίζεται στην υπόθεση ότι η αξιοπιστία του
σφάλματος ή το κόστος που δημιουργείται από το σφάλμα της πρόβλεψης, σχετίζεται
γραμμικά με το μέγεθος του σφάλματος.

\item \textbf{ Μέσο σφάλμα τετραγώνου MSE} (Mean Squared Error)\\
Το μέσο σφάλμα τετραγώνου ορίζεται ως το άθροισμα των τετραγώνων των
σφαλμάτων διαιρούμενο με τον αριθμό των χρονικών περιόδων n, στις οποίες έγιναν οι
προβλέψεις, δηλαδή:\\
$$ MSE= \frac{1}{n} \sum_{t=1}^n \left( Y_t-\widehat{Y}_t \right)^2 =\frac{1}{n} \sum_{t=1}^n  e_t^2 $$
Το MSE είναι η μέση τιμή των τετραγώνων των αποκλίσεων των προβλεπόμενων
τιμών της χρονοσειράς από τις αντίστοιχες πραγματικές. Η μονάδα μέτρησης του
MSE είναι εκφρασμένη στη μονάδα μέτρησης των τιμών των παρατηρήσεων
υψωμένη στο τετράγωνο.\\
Η ύπαρξη προβλέψεων που απέχουν πολύ από τις αντίστοιχες πραγματικές τιμές
γίνεται πολύ περισσότερο αισθητή με το κριτήριο MSE από ότι με το κριτήριο MAD,
επειδή οι τιμές των σφαλμάτων της πρόβλεψης υψώνονται στο τετράγωνο. Συνεπώς
το κριτήριο MSE είναι στατιστικά περισσότερο αξιόπιστο από το κριτήριο MAD και
χρησιμοποιείται συχνότερα για την επιλογή της ‘κατάλληλης’ μεθόδου πρόβλεψης.

\item \textbf{Ρίζα μέσου
σφάλματος τετραγώνου RMSE} (Root Mean Squared Error)\\
H τετραγωνική ρίζα μέσου
σφάλματος τετραγώνου είναι η
θετική τιμή της τετραγωνικής του ρίζας, δηλαδή είναι:\\
$$ RMSE=\sqrt{MSE}=\sqrt{\frac{1}{n}\sum_{t=1}} $$
Το RMSE εκφράζεται στην ίδια μονάδα μέτρησης με εκείνη των τιμών της
χρονοσειράς.\\


\item \textbf{Μέσο απόλυτο ποσοστιαίο σφάλμα MAPE} (Mean Absolute Percentage
Error)\\
Το μέσο απόλυτο ποσοστιαίο σφάλμα εξετάζει τη συμπεριφορά της απόλυτης
τιμής του σφάλματος της πρόβλεψης σε σχέση με την πραγματική τιμή της
χρονοσειράς. Το MAPE ορίζεται ως το άθροισμα των απόλυτων τιμών των
σφαλμάτων της πρόβλεψης προς τις αντίστοιχες πραγματικές τιμές της χρονοσειράς
διαιρούμενο με τον αριθμό των χρονικών περιόδων n, στις οποίες έγιναν προβλέψεις,
δηλαδή:\\
$$MAPE= \frac{1}{n} \sum_{t=1}^n \frac{\vert Y_t-\widehat{Y}_t \vert}{Y_t}=\frac{1}{n}\sum_{t=1}^n \frac{\vert e_t \vert}{Y_t} $$
Το κριτήριο αυτό είναι απαλλαγμένο από μονάδες μέτρησης και το
χρησιμοποιούμε για να συγκρίνουμε την ακρίβεια μιας ή περισσοτέρων μεθόδων
προβλέψεων και για περισσότερες από μια χρονοσειρές.

\item \textbf{Μέσο ποσοστιαίο σφάλμα MPE} (Mean Percentage Error)\\
Το μέσο ποσοστιαίο σφάλμα το χρησιμοποιούμε όταν ενδιαφερόμαστε να
προσδιορίσουμε αν η μέθοδος πρόβλεψης είναι μεροληπτική, δηλαδή αν οι
προβλεπόμενες τιμές είναι συστηματικά μεγαλύτερες ή μικρότερες από τις
αντίστοιχες πραγματικές.\\
$$ MPE= \frac{1}{n} \sum_{t=1}^n \frac{Y_t-\widehat{Y}_t}{Y_t} =\frac{1}{n} \sum_{t=1}^n \frac{e_t}{Y_t}$$
Χωρίς αμφιβολία, όσο πιο κοντά στο μηδέν είναι η τιμή του MPE, τόσο πιο
αμερόληπτη και καλή είναι η μέθοδος πρόβλεψης που χρησιμοποιήθηκε. Αντιθέτως,
μεγάλες απόλυτες τιμές του MPE φανερώνουν μεγάλη μεροληψία της μεθόδου.


\end{itemize}


%%%%%%%%%%%%%%%%%%%%%%%%%%%%%%%%%%%%%%%%%%%%%%%
\section{ΜΕΘΟΔΟΣ ΕΞΟΜΑΛΥΝΣΗΣ}
%%%%%%%%%%%%%%%%%%%%%%%%%%%%%%%%%%%%%%%%%%%%%%%


%%%%%%%%%%%%%%%%%%%%%%%%%%%%%%%%%%%%%%%%%%%%%%%
\subsection{ΑΠΛΟΣ ΚΙΝΗΤΟΣ ΜΕΣΟΣ}
%%%%%%%%%%%%%%%%%%%%%%%%%%%%%%%%%%%%%%%%%%%%%%%
\subsection{ ΑΠΛΗ ΕΚΘΕΤΙΚΗ ΕΞΟΜΑΛΥΝΣΗ}

\subsection{ΔΙΠΛΟΣ ΚΙΝΗΤΟΣ ΜΕΣΟΣ}
\subsection{ΜΕΘΟΔΟΣ BROWN}
\subsection{ ΜΕΘΟΔΟΣ HOLT}
\subsection{ΜΕΘΟΔΟΣ WINTERS}
%%%%%%%%%%%%%%%%%%%%%%%%%%%%%%%%%%%%%%%%%%%%%%%
\section{ΜΕΘΟΔΟΣ ΔΙΑΣΠΑΣΗΣ ΧΡΟΝΟΣΕΙΡΩΝ}
%%%%%%%%%%%%%%%%%%%%%%%%%%%%%%%%%%%%%%%%%%%%%%%
\subsection{ΑΝΑΛΥΣΗ ΠΟΧΙΚΟΤΗΤΑΣ }
\subsection{ΑΝΑΛΥΣΗ ΜΑΚΡΟΧΡΟΝΙΑΣ ΤΑΣΗΣ}
\subsection{ΑΝΑΛΥΣΗ ΚΥΚΛΙΚΟΤΗΤΑΣ ΚΑΙ ΜΗ ΚΑΝΟΝΙΚΟΤΗΤΑΣ}







%%%%%%%%%%%%%%% File ends here %%%%%%%%%%%%%%%%%%%%%%%%%%%%%%%%
\endinput
%%% Local Variables: 
%%% mode: latex
%%% TeX-master: "ptyxiakn"
%%% End: 
