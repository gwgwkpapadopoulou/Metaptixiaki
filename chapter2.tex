
%%%%%%%%%%%%%%%%%%%%%%%%%%%%%%%%%%%%%%%%%%%%%%%%%
\chapter{ΜΕΘΟΔΟΙ ΠΡΟΒΛΕΨΗΣ}
%%%%%%%%%%%%%%%%%%%%%%%%%%%%%%%%%%%%%%%%%%%%%%%%%
\section{ ΚΡΙΤΗΡΙΑ ΑΞΙΟΛΟΓΗΣΗΣ ΤΩΝ ΜΕΘΟΔΩΝ ΠΡΟΒΛΕΨΗΣ}
Η ανάλυση χρονοσειρών (time series analysis) ασχολείται αποκλειστικά με τη
διερεύνηση της διαχρονικής συμπεριφοράς των τιμών μιας μεταβλητής, οι
παρατηρήσεις της οποίας προέρχονται από χρονοσειρά. Η πρόβλεψη των
μελλοντικών τιμών της μεταβλητής σύμφωνα με την ανάλυση χρονοσειρών μπορεί
να προέλθει απο διάφορες κατηγορίες μεθόδων προβλέψης, όπως η μέθοδος Εξομάλυνσης και η Διάσπαση των χρονοσειρών, με τις οποίες θα ασχοληθούμε στην παρούσα εργασία.\\
Για την επιλογή της κατάλληλης μεθόδου χρησιμοποιούνται τα κριτήρια
αξιολόγησης των μεθόδων προβλέψεων. Τα κριτήρια αυτά βασίζονται στις τιμές των
αποκλίσεων των προβλεπόμενων τιμών από τις αντίστοιχες πραγματικές τιμές της
χρονοσειράς.\\
Για μία μεταβλητή Y, η απόκλιση της προβλεπόμενης τιμής της $ \widehat{Y}_t $  από την
αντίστοιχη πραγματική τιμή της $Y_t$ για την περίοδο t, όπου $t=1,2,3,\ldots,n , \:$ ονομάζεται
σφάλμα της πρόβλεψης (forecast error), συμβολίζεται με $e_t$ και ορίζεται ως:
$\: e_t = Y_t - \widehat{Y}_t$\\
Η παραπάνω σχέση εκφράζει για κάθε περίοδο t τη διαφορά μεταξύ της πραγματικής
τιμής $Y_t$ και της αντίστοιχης προβλεπόμενης τιμής $ \widehat{Y}_t $ που προήλθε από τη μέθοδο
πρόβλεψης που χρησιμοποιήθηκε.\\
Επομένως, για να προσδιορίσουμε την αξιοπιστία μιας συγκεκριμένης μεθόδου
πρόβλεψης, θα πρέπει να μελετήσουμε τη διαχρονική συμπεριφορά των τιμών των
σφαλμάτων της πρόβλεψης. Αυτό γίνεται με την εφαρμογή διάφορων κριτηρίων,
σύμφωνα με τα οποία αξιολογούμε τη χρησιμοποιούμενη μέθοδο πρόβλεψης. Κάθε
ένα από τα κριτήρια αυτά ορίζεται από μία συγκεκριμένη συναρτησιακή σχέση των
σφαλμάτων της πρόβλεψης και μπορεί να χρησιμοποιηθεί όχι μόνο για την
αξιολόγηση μιας μεθόδου πρόβλεψης αλλά και για την επιλογή της “ καλύτερης ”
μεταξύ δύο ή περισσοτέρων εναλλακτικών μεθόδων προβλέψεων. Τα κριτήρια αυτά
είναι:\\
\begin{itemize}
\item \textbf{Μέση απόλυτη απόκλιση MAD} (Mean Absolute Deviation)\\
Η μέση απόλυτη απόκλιση ορίζεται ως το άθροισμα των απόλυτων τιμών του
σφάλματος της πρόβλεψης διαιρούμενο με τον αριθμό των περιόδων n, στις οποίες
έγιναν προβλέψεις, δηλαδή:\\
$$ MAD= \frac{1}{n} \sum_{t=1}^n \vert Y_t-\widehat{Y}_t \vert =\frac{1}{n} \sum_{t=1}^n \vert e_t \vert $$
Το MAD εκφράζει τη μέση τιμή των απολύτων αποκλίσεων των προβλεπόμενων
τιμών της χρονοσειράς από τις αντίστοιχες πραγματικές και έχει τα ακόλουθα
χαρακτηριστικά. Πρώτον, η μονάδα μέτρησης του είναι η ίδια με εκείνη των τιμών
της χρονοσειράς και έτσι είναι εύκολη η ερμηνεία του. Δεύτερον, στον υπολογισμό
του λαμβάνονται υπ’ όψιν μόνο οι απόλυτες τιμές των σφαλμάτων και όχι οι
πραγματικές τιμές τους. Αυτό σημαίνει ότι το MAD είναι ανεξάρτητο από θετικές ή αρνητικές τιμές του σφάλματος, δηλαδή είναι ανεξάρτητο από το αν οι τιμές των
προβλέψεων είναι μικρότερες ή μεγαλύτερες των
πραγματικών τιμών.Και τέλος, το MAD βασίζεται στην υπόθεση ότι η αξιοπιστία του
σφάλματος ή το κόστος που δημιουργείται από το σφάλμα της πρόβλεψης, σχετίζεται
γραμμικά με το μέγεθος του σφάλματος.

\item \textbf{ Μέσο σφάλμα τετραγώνου MSE} (Mean Squared Error)\\
Το μέσο σφάλμα τετραγώνου ορίζεται ως το άθροισμα των τετραγώνων των
σφαλμάτων διαιρούμενο με τον αριθμό των χρονικών περιόδων n, στις οποίες έγιναν οι
προβλέψεις, δηλαδή:\\
$$ MSE= \frac{1}{n} \sum_{t=1}^n \left( Y_t-\widehat{Y}_t \right)^2 =\frac{1}{n} \sum_{t=1}^n  e_t^2 $$
Το MSE είναι η μέση τιμή των τετραγώνων των αποκλίσεων των προβλεπόμενων
τιμών της χρονοσειράς από τις αντίστοιχες πραγματικές. Η μονάδα μέτρησης του
MSE είναι εκφρασμένη στη μονάδα μέτρησης των τιμών των παρατηρήσεων
υψωμένη στο τετράγωνο.\\
Η ύπαρξη προβλέψεων που απέχουν πολύ από τις αντίστοιχες πραγματικές τιμές
γίνεται πολύ περισσότερο αισθητή με το κριτήριο MSE από ότι με το κριτήριο MAD,
επειδή οι τιμές των σφαλμάτων της πρόβλεψης υψώνονται στο τετράγωνο. Συνεπώς
το κριτήριο MSE είναι στατιστικά περισσότερο αξιόπιστο από το κριτήριο MAD και
χρησιμοποιείται συχνότερα για την επιλογή της ‘κατάλληλης’ μεθόδου πρόβλεψης.

\item \textbf{Ρίζα μέσου
σφάλματος τετραγώνου RMSE} (Root Mean Squared Error)\\
H τετραγωνική ρίζα μέσου
σφάλματος τετραγώνου είναι η
θετική τιμή της τετραγωνικής του ρίζας, δηλαδή είναι:\\
$$ RMSE=\sqrt{MSE}=\sqrt{\frac{1}{n}\sum_{t=1}^n e_t^2} $$
Το RMSE εκφράζεται στην ίδια μονάδα μέτρησης με εκείνη των τιμών της
χρονοσειράς.\\


\item \textbf{Μέσο απόλυτο ποσοστιαίο σφάλμα MAPE} (Mean Absolute Percentage
Error)\\
Το μέσο απόλυτο ποσοστιαίο σφάλμα εξετάζει τη συμπεριφορά της απόλυτης
τιμής του σφάλματος της πρόβλεψης σε σχέση με την πραγματική τιμή της
χρονοσειράς. Το MAPE ορίζεται ως το άθροισμα των απόλυτων τιμών των
σφαλμάτων της πρόβλεψης προς τις αντίστοιχες πραγματικές τιμές της χρονοσειράς
διαιρούμενο με τον αριθμό των χρονικών περιόδων n, στις οποίες έγιναν προβλέψεις,
δηλαδή:\\
$$MAPE= \frac{1}{n} \sum_{t=1}^n \frac{\vert Y_t-\widehat{Y}_t \vert}{Y_t}=\frac{1}{n}\sum_{t=1}^n \frac{\vert e_t \vert}{Y_t} $$
Το κριτήριο αυτό είναι απαλλαγμένο από μονάδες μέτρησης και το
χρησιμοποιούμε για να συγκρίνουμε την ακρίβεια μιας ή περισσοτέρων μεθόδων
προβλέψεων και για περισσότερες από μια χρονοσειρές.

\item \textbf{Μέσο ποσοστιαίο σφάλμα MPE} (Mean Percentage Error)\\
Το μέσο ποσοστιαίο σφάλμα το χρησιμοποιούμε όταν ενδιαφερόμαστε να
προσδιορίσουμε αν η μέθοδος πρόβλεψης είναι μεροληπτική, δηλαδή αν οι
προβλεπόμενες τιμές είναι συστηματικά μεγαλύτερες ή μικρότερες από τις
αντίστοιχες πραγματικές.\\
$$ MPE= \frac{1}{n} \sum_{t=1}^n \frac{Y_t-\widehat{Y}_t}{Y_t} =\frac{1}{n} \sum_{t=1}^n \frac{e_t}{Y_t}$$
Χωρίς αμφιβολία, όσο πιο κοντά στο μηδέν είναι η τιμή του MPE, τόσο πιο
αμερόληπτη και καλή είναι η μέθοδος πρόβλεψης που χρησιμοποιήθηκε. Αντιθέτως,
μεγάλες απόλυτες τιμές του MPE φανερώνουν μεγάλη μεροληψία της μεθόδου.


\end{itemize}


%%%%%%%%%%%%%%%%%%%%%%%%%%%%%%%%%%%%%%%%%%%%%%%
\section{ΜΕΘΟΔΟΙ ΕΞΟΜΑΛΥΝΣΗΣ}
%%%%%%%%%%%%%%%%%%%%%%%%%%%%%%%%%%%%%%%%%%%%%%%

Οι μέθοδοι εξομάλυνσης (smoothing methods) είναι τεχνικές με τις οποίες
προσδιορίζονται οι μελλοντικές τιμές μιας μεταβλητής με βάση τον τρόπο εφαρμογής
τους. Οι τεχνικές αυτές ονομάζονται μέθοδοι εξομάλυνσης, διότι η δημιουργία των
προβλέψεων προέρχεται από την εξομάλυνση της διαχρονικής εξέλιξης των τιμών της
μεταβλητής, ώστε να αναγνωριστεί καλύτερα ο τρόπος συμπεριφοράς της. Ορισμένες
από τις μεθόδους εξομάλυνσης μπορούν να εφαρμοστούν και σε περιπτώσεις μικρού
αριθμού παρατηρήσεων της μεταβλητής. Οι μέθοδοι εξομάλυνσης που θα
περιγράψουμε είναι:\\
\begin{itemize}
\item Η μέθοδος του απλού κινητού μέσου
\item Η μέθοδο της απλής εκθετικής εξομάλυνσης
\item Η μέθοδος του διπλού κινητού μέσου
\item Η μέθοδος Brown
\item Η μέθοδος Holt
\item Η μέθοδος Winters
\end{itemize}

Εάν μία χρονοσειρά είναι στάσιμη η κατάλληλη μέθοδος πρόβλεψης μελλοντικών
τιμών είναι η μέθοδος των κινητών μέσων όρων. Σε μερικές χρονοσειρές όμως οι
πρόσφατες παρατηρήσεις μπορεί να περιέχουν περισσότερες πληροφορίες από τις
παλαιότερες και αυτό είναι πολύ σημαντικό για τις μελλοντικές προβλέψεις. Σε αυτήν
την περίπτωση χρησιμοποιούμε την απλή εκθετική εξομάλυνση. Εάν η χρονοσειρά
εμφανίζει κάποιο πρότυπο τάσης τότε χρησιμοποιούμε την μέθοδο της διπλής
εκθετικής εξομάλυνσης, την μέθοδο Brown ή την μέθοδο Holt ενώ εάν η χρονοσειρά
εμφανίζει εποχικότητα τότε χρησιμοποιούμε την μέθοδο Winters.

%%%%%%%%%%%%%%%%%%%%%%%%%%%%%%%%%%%%%%%%%%%%%%%
\subsection{ΑΠΛΟΣ ΚΙΝΗΤΟΣ ΜΕΣΟΣ}
%%%%%%%%%%%%%%%%%%%%%%%%%%%%%%%%%%%%%%%%%%%%%%%
Η μέθοδος του απλού κινητού μέσου m-περιόδων (simple moving average) είναι
μία πολύ απλή μέθοδος προβλέψεων που χρησιμοποιεί ως πρόβλεψη την τιμή του
αριθμητικού μέσου όρου των m πλέον πρόσφατων παρατηρήσεων της χρονοσειράς.
Αυτό συμβαίνει διότι οι πλέον πρόσφατες παρατηρήσεις της χρονοσειράς θεωρούνται
περισσότερο αντιπροσωπευτικές για τη δημιουργία προβλέψεων από ότι οι πιο
απομακρυσμένες στο παρελθόν. Ο μέσος όρος αυτός ονομάζεται κινητός, επειδή η
τιμή του δεν είναι σταθερή, αλλά αναπροσαρμόζεται κάθε φορά που μια νέα
παρατήρηση της χρονοσειράς γίνεται διαθέσιμη.\\
Οι προβλέψεις μιας χρονοσειράς $Y_t$ , για $t=1,2,\ldots,n \:\:$, δημιουργούνται με τη μέθοδο
του απλού κινητού μέσου ως εξής:\\
$$ \widehat{Y}_{t+1}=\frac{1}{m}\sum_{j=1}^m Y_{t-j+1}=\frac{1}{m} \left(Y_t +Y_{t-1}+\ldots+Y_{t-m+1}\right)=\widehat{Y}_t +\frac{Y_t}{m}-\frac{Y_{t-m}}{m} $$
όπου $\widehat{Y}_{t+1} $ είναι η πρόβλεψη για την περίοδο (t+1) και m ο αριθμός των περιόδων που
χρησιμοποιούνται για τον υπολογισμό της τιμής του μέσου όρου. Επίσης για m=1 η
πρόβλεψη της επόμενης περιόδου είναι ίση με την πραγματική τιμή της
προηγούμενης περιόδου, δηλαδή ισχύει η σχέση:
$\: \widehat{Y}_{t+1}=Y_t $ \\

Συνήθως για να προσδιορίσουμε την τιμή του m για τη δημιουργία προβλέψεων
σε μια χρονοσειρά, εφαρμόζουμε τη μέθοδο του απλού κινητού μέσου στη
χρονοσειρά για διαφορετικές τιμές του m και επιλέγουμε εκείνη την τιμή του m που
ελαχιστοποιεί την τιμή του κριτηρίου MSE ή κάποιου άλλου κριτηρίου.
%%%%%%%%%%%%%%%%%%%%%%%%%%%%%%%%%%%%%%%%%%%%%
\subsection{ ΑΠΛΗ ΕΚΘΕΤΙΚΗ ΕΞΟΜΑΛΥΝΣΗ}
%%%%%%%%%%%%%%%%%%%%%%%%%%%%%%%%%%%%%%%%%%%%%
Ένα μειονέκτημα της μεθόδου του απλού κινητού μέσου m-περιόδων είναι ότι για
τον υπολογισμό των προβλέψεων δίνει ίση βαρύτητα σε κάθε παρατήρηση,
ανεξάρτητα από το πόσο κοντά ή μακριά βρίσκεται σε σχέση με την προβλεπόμενη
περίοδο. Το μειονέκτημα αυτό μπορεί να εξαλειφθεί με τη μέθοδο της απλής
εκθετικής εξομάλυνσης (simple exponential smoothing), σύμφωνα με την οποία οιπροβλέψεις δημιουργούνται με βάση κάποιο σταθμικό μέσο όρο, έτσι ώστε να δίνεται
διαφορετική βαρύτητα σε κάθε παρατήρηση. Πιο συγκεκριμένα, με τη μέθοδο αυτή
δίνεται πολύ μεγαλύτερη βαρύτητα στις πιο πρόσφατες παρατηρήσεις, από αυτή που
δίνεται στις πιο απομακρύσμενες.\\
Για να κατανοήσουμε το μηχανισμό λειτουργίας της μεθόδου ας θεωρήσουμε ότι
οι προβλέψεις της χρονοσειράς δημιουργούνται ως εξής:\\
\begin{equation}
\label{ekthetikh}
 \widehat{Y}_{t + 1} = aY_t + a \left( 1 − a \right) Y_{t − 1} + a \left( 1 − a \right)^2 Y_{t − 2} + \dots 
\end{equation}
που η παράμετρος α ονομάζεται σταθερά εξομάλυνσης (smoothing constant) και
λαμβάνει τιμές μεταξύ 0 και 1 δηλαδή: $ \: 0\leq a\leq 1$\\

Έτσι, σύμφωνα με την παραπάνω σχέση η πρόβλεψη $\widehat{Y}_{t+1} $ προκύπτει ως ένας
σταθμικός μέσος όρος των παρατηρήσεων της χρονοσειράς, αφού το άθροισμα των
συντελεστών της σχέσης $\left(\ref{ekthetikh}\right) $ είναι ίσο με τη μονάδα. Όσο πιο μεγάλη είναι η τιμή της
παραμέτρου α, τόσο μεγαλύτερη βαρύτητα δίνεται στις πιο πρόσφατες παρατηρήσεις
και πολύ μικρή εως μηδαμινή βαρύτητα στις πιο απομακρυσμένες.\\
Η παραπάνω σχέση μπορεί να γραφεί και με τη μορφή:\\
$$ \widehat{Y}_{t + 1} = aY_t + \left( 1 − a \right) \widehat{Y}_t $$
Η σχέση αυτή είναι η μαθηματική έκφραση της μεθόδου της απλής εκθετικής
εξομάλυνσης και ορίζεται για $\:t=2,3,\ldots,n\:$ με αρχική συνθήκη $\widehat{Y}_2 = Y_1$ .

Η τιμή της παραμέτρου α καθορίζεται από τον ερευνητή. Ωστόσο, πιο
αντικειμενικό είναι η “άριστη” τιμή του α να προσδιορίζεται από τα δεδομένα της
χρονοσειράς. Ειδικότερα, εφαρμόζοντας τη μέθοδο της απλής εκθετικής εξομάλυνσης
στις παρατηρήσεις της χρονοσειράς, για τιμές του α από το μηδέν μέχρι τη μονάδα
επιλέγουμε εκείνη την τιμή του α που ελαχιστοποιεί την τιμή του κριτηρίου MSE ή
κάποιου άλλου κριτηρίου.

\subsection{ΔΙΠΛΟΣ ΚΙΝΗΤΟΣ ΜΕΣΟΣ}

%%%%%%%%%%%%%%%%%%%%%%%%%%%%%%%%%%
\subsection{ΜΕΘΟΔΟΣ BROWN}
%%%%%%%%%%%%%%%%%%%%%%%%%%%%%%%%%
Η μέθοδος της διπλής εκθετικής εξομάλυνσης (double exponential smoothing), η
οποία ονομάζεται και μέθοδος Brown, είναι μια άλλη μέθοδος προβλέψεων που
χρησιμοποιείται σε χρονοσειρές, οι παρατηρήσεις των οποίων παρουσιάζουν τάση. Η
βασική φιλοσοφία της μεθόδου αυτής είναι παραπλήσια με εκείνη της μεθόδου του
διπλού κινητού μέσου, δηλαδή η εξομάλυνση των παρατηρήσεων της χρονοσειράς
γίνεται δύο φορές, ενώ στη διαμόρφωση των προβλέψεων λαμβάνεται υπ’όψιν και η
τάση.\\

Η εφαρμογή της μεθόδου της διπλής εκθετικής εξομάλυνσης στηρίζεται στην
ακόλουθη διαδικασία:\\
\begin{enumerate}
\item  Εξομαλύνουμε τις αρχικές παρατηρήσεις της χρονοσειράς με τη μέθοδο της απλής
εκθετικής εξομάλυνσης ως εξής:\\
$$ A_t=aY_t + \left(1-a\right) A_{t-1} $$
όπου α είναι η σταθερά εξομάλυνσης, για $ 0 \leq a \leq 1 $ , $ A_t $ οι εξομαλυνθείσες τιμές της
χρονοσειράς που προκύπτουν από την πρώτη εξομάλυνση, για $ t = 2,3,\ldots,n $ , ενώ για
$t=1$ ορίζεται ως αρχική συνθήκη $ A_1 = Y_1 $ .

\item Εξομαλύνουμε τις εξομαλυνθείσες τιμές $A_t$ της χρονοσειράς με τη μέθοδο της
απλής εκθετικής εξομάλυνσης ως εξής:\\
$$ A^{'}_t=aA_t + \left(1-a\right) A^{'}_{t-1} $$
όπου $A^{'}_t$ είναι οι εξομαλυνθείσες τιμές της χρονοσειράς που προκύπτουν από τη
δεύτερη εξομάλυνση, για $t = 2,3,\dots,n\:$ ενώ για $t=1$, $A^{'}_1 =A_1$.

\item Υπολογίζουμε τη διαφορά $a_t$ ως εξής:
$$a_t = 2 A_t – A^{'}_t ́$$

\item Υπολογίζουμε τον παράγοντα προσαρμογής για την τάση, $b_t$, ως εξής:\\
$$ b_t=\frac{a}{1-a} \left(A_t - A^{'}_t \right) $$

\item Υπολογίζουμε την πρόβλεψη $ \widehat{Y}_{t+h} $ για την h μελλοντική περίοδο ως εξής:\\
$$ \widehat{Y}_{t+h}=a_t+hb_t  $$
όπου h είναι ένας ακέραιος θετικός αριθμός.\\

Η μέθοδος αυτή μπορεί να εφαρμοστεί για τη διαμόρφωση προβλέψεων για
περισσότερες από μία μελλοντικές περιόδους σε αντίθεση με τη μέθοδο της απλής
εκθετικής εξομάλυνσης, η οποία παρέχει προβλέψεις μόνο για την επόμενη χρονική
περίοδο. Επίσης, αν η τιμή της σταθεράς εξομάλυνσης $a$ δεν είναι γνωστή, κάτι που
συμβαίνει όταν εφαρμόζουμε τη μέθοδο για πρώτη φορά στα δεδομένα μιας
χρονοσειράς, επιλέγουμε κατά τα γνωστά εκείνη την τιμή του $a$ που ελαχιστοποιεί
την τιμή του κριτηρίου MSE ή κάποιου άλλου κριτηρίου. Σημειώνουμε ότι ο αριθμός
των παρατηρήσεων που απαιτούνται για την εφαρμογή της μεθόδου αυτής είναι αρκετά μικρότερος από τον αντίστοιχο αριθμό της μεθόδου του διπλού κινητού
μέσου.
 
\end{enumerate}
%%%%%%%%%%%%%%%%%%%%%%%%%%%%%%%%%%%%%%%%%%%
\subsection{ ΜΕΘΟΔΟΣ HOLT}
%%%%%%%%%%%%%%%%%%%%%%%%%%%%%%%%%%%%%%%%%%%%


\subsection{ΜΕΘΟΔΟΣ WINTERS}
%%%%%%%%%%%%%%%%%%%%%%%%%%%%%%%%%%%%%%%%%%%%%%%
\section{ΜΕΘΟΔΟΣ ΔΙΑΣΠΑΣΗΣ ΧΡΟΝΟΣΕΙΡΩΝ}
%%%%%%%%%%%%%%%%%%%%%%%%%%%%%%%%%%%%%%%%%%%%%%%
\subsection{ΑΝΑΛΥΣΗ ΠΟΧΙΚΟΤΗΤΑΣ }
\subsection{ΑΝΑΛΥΣΗ ΜΑΚΡΟΧΡΟΝΙΑΣ ΤΑΣΗΣ}
\subsection{ΑΝΑΛΥΣΗ ΚΥΚΛΙΚΟΤΗΤΑΣ ΚΑΙ ΜΗ ΚΑΝΟΝΙΚΟΤΗΤΑΣ}







%%%%%%%%%%%%%%% File ends here %%%%%%%%%%%%%%%%%%%%%%%%%%%%%%%%
\endinput
%%% Local Variables: 
%%% mode: latex
%%% TeX-master: "ptyxiakn"
%%% End: 
